\documentclass[leqno]{report}
\usepackage[utf8]{inputenc}
\usepackage[T1]{fontenc}
\usepackage{geometry}
\RequirePackage[orthodox]{nag}
\usepackage{microtype}
\usepackage{booktabs}
\usepackage{mathptmx}
\usepackage{mathtools}
\usepackage{amssymb}
\usepackage{amsthm}
\usepackage{stmaryrd}
\usepackage{booktabs}
\usepackage{enumerate}
\usepackage{multirow}
\usepackage{nicefrac}
\usepackage{xspace}

\usepackage{algpseudocode}
\usepackage[colorlinks=true,linkcolor=blue]{hyperref}
\usepackage[colorinlistoftodos]{todonotes}

\allowdisplaybreaks

\theoremstyle{definition}
\newtheorem{Definition}{Definition}

\theoremstyle{plain}
\newtheorem{Thm}{Theorem}
\newtheorem{Conjecture}{Conjecture}
\newtheorem{Corollary}[Thm]{Corollary}
\newtheorem{Example}[Thm]{Example}
\newtheorem{Lemma}[Thm]{Lemma}
\newtheorem{Proposition}[Thm]{Proposition}
\newtheorem{Question}{Question}
\newtheorem{Remark}[Thm]{Remark}
\newtheorem{Theorem}[Thm]{Theorem}

\renewcommand{\mid}{\, \middle| \,}
\newcommand{\abs}[1]{\left| #1 \right|}
\newcommand{\menge}[1]{\ensuremath{\left\{ #1 \right\}}}
\newcommand{\set}[2]{\mbox{$ \left\{ \,#1 \mid #2 \,\right\}$}}
\newcommand{\ext}[1]{\llbracket #1  \rrbracket}
\newcommand{\fa}[2]{\forall {#1} \quad {#2}}
\newcommand{\ex}[2]{\exists {#1} \quad {#2}}

\DeclarePairedDelimiter\floor{\lfloor}{\rfloor}

\newcommand{\N}{\mathbb{N}}
\newcommand{\Q}{\mathbb{Q}}
\newcommand{\R}{\mathbb{R}}

\newcommand{\FCCS}{\textsc{fccs}}

\begin{document}
\author{Chris Wong}
\title{Properties of fast converging Cauchy sequences}
\maketitle

Insert introduction here.

\begin{Definition}
    A fast converging Cauchy sequence (\FCCS) $x : \N \rightarrow \Q$ has the property that
    \[ \fa{m, n \in \N}{ \abs{x_m - x_{m+n}} < \frac{1}{2^m} } \ . \]
\end{Definition}

\section{Proofs}

\begin{Proposition}[Addition]
    \label{add}
    Let $x, y$ be \FCCS.

    The sum $x + y$, defined by $(x+y)_k \mapsto x_{k+1} + y_{k+1}$, is a \FCCS.
\end{Proposition}

\begin{proof}
    Let $m, n \in \N$ be arbitrary.

    Then
    \begin{align*}
        \abs{(x+y)_m - (x+y)_{m+n}}
        &= \abs{(x_{m+1} + y_{m+1}) - (x_{m+n+1} + y_{m+n+1})} \\
        &= \abs{x_{m+1} - x_{m+n+1} + y_{m+1} - y_{m+n+1}} \\
        &\leq \abs{x_{m+1} - x_{m+n+1}} + \abs{y_{m+1} - y_{m+n+1}} \\
        &< \frac{1}{2^{m+1}} + \frac{1}{2^{m+1}} \\
        &= \frac{1}{2^m}
        \ . \qedhere
    \end{align*}
\end{proof}

\begin{Proposition}[Negation]
    \label{neg}
    If $x$ is a \FCCS, then $(-x)_k \mapsto -(x_k)$ is a \FCCS.
\end{Proposition}

\begin{proof}
    Let $m, n \in \N$ be arbitrary.

    Then
    \[
        \abs{(-x)_m - (-x)_{m+n}}
        = \abs{-x_m - (-x_{m+n})}
        = \abs{x_m - x_{m+n}}
        < \frac{1}{2^m}
        \ . \qedhere
    \]
\end{proof}

\begin{Corollary}[Subtraction]
    $(x-y)_k \mapsto x_{k+1} - y_{k+1}$ is a \FCCS.
\end{Corollary}

\begin{Proposition}[Multiplication]
    $(xy)_k \mapsto x_{a(k)} \cdot y_{b(k)}$ is a \FCCS, where
    \begin{align*}
        a(k) &= k + 1 + \phi(y_0) \\
        b(k) &= k + 1 + \phi(x_0) \\
        \phi(z) &= \max \menge{ 0,\, \left\lceil \log_2(\abs{z} + 1) \right\rceil }
        \ .
    \end{align*}
\end{Proposition}

\begin{proof}
    Given $m, n \in \N$, we need to choose $a,\, b : \N \rightarrow \N$ such that
    \begin{align*}
        \abs{(xy)_m - (xy)_{m+n}}
        &= \abs{x_{a(m)} y_{b(m)} - x_{a(m+n)} y_{b(m+n)}} \\
        &= \abs{x_{a(m)} y_{b(m)} - x_{a(m)} y_{b(m+n)} + x_{a(m)} y_{b(m+n)} - x_{a(m+n)} y_{b(m+n)}} \\
        &\leq \abs{x_{a(m)} y_{b(m)} - x_{a(m)} y_{b(m+n)}} + \abs{x_{a(m)} y_{b(m+n)} - x_{a(m+n)} y_{b(m+n)}} \\
        &= \abs{x_{a(m)}}\abs{y_{b(m)} - y_{b(m+n)}} + \abs{y_{b(m+n)}}\abs{x_{a(m)} - x_{a(m+n)}} \\
        &< \frac{1}{2^m}
        \ .
    \end{align*}

    For this inequality to hold, it is sufficient to show that
    \[ \abs{x_{a(m)}}\abs{y_{b(m)} - y_{b(m+n)}} < \frac{1}{2^{m+1}} \]
    and
    \[ \abs{y_{b(m+n)}}\abs{x_{a(m)} - x_{a(m+n)}} < \frac{1}{2^{m+1}} \ . \]

    To complete the proof, we observe that $\abs{x_{a(m)}} < \abs{x_0} + 1$ and $\phi(x_0) \geq \log_2(\abs{x_0} + 1)$. Using these facts:
    \begin{align*}
        \abs{x_{a(m)}}\abs{y_{b(m)} - y_{b(m+n)}}
        &< (\abs{x_0} + 1) \cdot \frac{1}{2^{b(m)}} \\
        &= \frac{\abs{x_0} + 1}{2^{m + 1 + \phi(x_0)}} \\
        &\leq \frac{\abs{x_0} + 1}{2^{m + 1 + \log_2(\abs{x_0} + 1)}} \\
        &= \frac{\abs{x_0} + 1}{2^{m + 1} \cdot (\abs{x_0} + 1)} \\
        &= \frac{1}{2^{m + 1}}
        \ .
    \end{align*}

    Similarly,
    \begin{align*}
        \abs{y_{a+b}}\abs{x_a - x_{a+b}}
        &< (\abs{y_0} + 1) \cdot \frac{1}{2^{m + 1 + \log_2(\abs{y_0} + 1)}} \\
        &= \frac{1}{2^{m + 1}}
        \ . \qedhere
    \end{align*}
\end{proof}

\begin{Definition}[Zero]
    A \FCCS \ $x$ is zero when
    \[ \fa{n \in \N}{\abs{x_n} \leq \frac{1}{2^n}} \ . \]
\end{Definition}

\begin{Lemma}[Apartness]
    Assume that Markov's principle is true. Then if $x$ is not zero, there exists $a \in \N$ where
    \[ \fa{n \in \N}{\abs{x_{c+n}} > \frac{1}{2^c}} \ . \]
\end{Lemma}

\begin{proof}
    If $x$ is not zero, then by Markov's principle
    \[
        \neg \left( \fa{n \in \N}{\abs{x_n} \leq \frac{1}{2^n}} \right)
        \quad \Rightarrow \quad
        \ex{n \in \N}{\abs{x_n} > \frac{1}{2^n}} \ .
    \]

    Applying the definition of a \FCCS, for all $k \in \N$
    \[
        \abs{x_n} - \abs{x_{n + k}}
        \leq \abs{x_n - x_{n + k}}
        < \frac{1}{2^n} \ .
    \]

    Rearranging we get
    \[
        \abs{x_n} - \frac{1}{2^n} < \abs{x_{n + k}}
    \]

    Now let $c \geq n$ be the smallest natural number such that
    \[
        \frac{1}{2^c} \leq \abs{x_n} - \frac{1}{2^n} < \abs{x_{n + k}}
    \]

    Then $\abs{x_{c + k}} > 1/2^c$ for all $k$.
\end{proof}

\begin{Proposition}[Reciprocal]
    $\left(x^{-1}\right)_k \mapsto \left(x_{a(k)}\right)^{-1}$ is a \FCCS, where TODO
\end{Proposition}

\begin{proof}
    TODO
\end{proof}

\bibliographystyle{abbrv}
\bibliography{All}

\end{document}

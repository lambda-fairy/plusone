\documentclass{article}
\usepackage[utf8]{inputenc}
\usepackage[T1]{fontenc}
\usepackage{geometry}
\RequirePackage[orthodox]{nag}
\usepackage{microtype}
\usepackage{booktabs}
\usepackage{mathtools}
\usepackage{amssymb}
\usepackage{amsthm}
\usepackage{stmaryrd}
\usepackage{csquotes}
\usepackage{enumitem}
\usepackage{nicefrac}
\usepackage{todonotes}
\usepackage{xspace}

\newcommand{\fa}[2]{\forall {#1} \quad {#2}}
\newcommand{\ex}[2]{\exists {#1} \quad {#2}}

% http://tex.stackexchange.com/a/23181/36565
\makeatletter
\MHInternalSyntaxOn
\newcommand*\DeclarePairedDelimiterAlt[3]{%
  \@ifdefinable{#1}{
    \@namedef{MT_delim_\MH_cs_to_str:N #1 _star:}##1
        {\mathopen{}\mathclose\bgroup\left#2 ##1 \aftergroup\egroup\right #3}%
    \@xp\@xp\@xp
      \newcommand
        \@xp\csname MT_delim_\MH_cs_to_str:N #1 _nostar:\endcsname
        [2][\\@gobble]
        {
          \mathopen{\@nameuse {\MH_cs_to_str:N ##1 l} #2} ##2
          \mathclose{\@nameuse {\MH_cs_to_str:N ##1 r} #3}}
    \DeclareRobustCommand{#1}{
      \@ifstar
        {\@nameuse{MT_delim_\MH_cs_to_str:N #1 _nostar:}}
        {\@nameuse{MT_delim_\MH_cs_to_str:N #1 _star:}}
    }
  }
}
\MHInternalSyntaxOff
\makeatother

\DeclarePairedDelimiterAlt\abs{\lvert}{\rvert}
\DeclarePairedDelimiterAlt\ceil{\lceil}{\rceil}
\DeclarePairedDelimiterAlt\floor{\lfloor}{\rfloor}
\DeclarePairedDelimiterAlt\menge{\{}{\}}

\newcommand{\N}{\mathbb{N}}
\newcommand{\Q}{\mathbb{Q}}
\newcommand{\R}{\mathbb{R}}

\begin{document}
\author{Chris Wong}
\title{The Ramsey number $R(5,5)$}
\maketitle

Hello, world!

\section{Introduction}

Before we introduce the Ramsey numbers, we must first define a few terms.

A \textit{graph} $G = (V, E)$ consists of a set of \textit{vertices} $V$ connected by a set of \textit{edges} $E$. For the purposes of this paper, we assume that the graphs are \textit{simple}: that is, every pair of vertices has at most one edge between them.

A graph is \textit{complete} when every pair of vertices is connected by an edge. For each positive integer $n$, the complete graph on $n$ vertices is unique and denoted $K_n$.

An \textit{(edge-induced) subgraph} $G' \subseteq G$ is formed from a subset of the edges of $G$, along with the vertices incident to them. When a subgraph is complete, it is called a \textit{clique}.

An \textit{(edge) coloring} labels every edge of a graph by elements from a given set of colors. By tradition, we use the colors \textit{red} and \textit{blue}.

With these definitions, we can state Ramsey's theorem, and with it, the Ramsey numbers:

\begin{Theorem}[Ramsey's theorem] \label{ramseys_theorem}
    Let $s$ and $t$ be positive integers. Then for sufficiently large $n$, every red-blue edge coloring of $K_n$ either has a fully red clique of size $s$, or a fully blue clique of size $t$.
\end{Theorem}

\begin{Definition}[Ramsey numbers]
    Define $R(s,t)$ as the smallest $n$ such that the conclusion of \cref{ramseys_theorem} holds. This is the \textit{Ramsey number} of $(s,t)$.
\end{Definition}

\section{Computing Ramsey numbers}

The definition of Ramsey numbers is easy to state. What is surprising, then, is how difficult they are to compute. As of this writing, the last exact result was proven by \cite{JGT:JGT3190190304}, who showed that $R(4,5) = 25$. The value of $R(5,5)$---just one step above---is still an open problem!

This difficulty stems from the sheer number of combinations that must be checked. Since a Ramsey number is a statement about \emph{all} colorings of a particular graph, it is not enough to find a single example---we have to account for them all. Moreover, as the number of possible colorings grows exponentially with the number of vertices, it is almost never feasible to check every coloring. In fact, there is no known polynomial-time algorithm for computing Ramsey numbers in general; mathematicians so far have used a separate, ad-hoc proof for each result.

\section{Upper and lower bounds}

While it can be difficult to compute a Ramsey number exactly, we can often derive bounds on what values the number can take. These bounds can be an important step toward finding the exact result. In particular, if we can show that $R(s,t) \geq n$ and $R(s,t) \leq n$ for the same $n$, then it follows that $R(s,t) = n$.

To prove a lower bound for a Ramsey number---that is, $R(s,t) > n$---we only need a counterexample: a coloring on $K_n$ which does \emph{not} have a red clique of size $s$ or a blue clique of size $t$. This coloring is called a \textit{Ramsey graph} of size $n$. Since a Ramsey number is a statement about all colorings on $K_n$, such a coloring is sufficient to show that $R(s,t) \nleq n$, hence $R(s,t) > n$.

On the other hand, to prove an upper bound is often more difficult. This is because to show that $R(s,t) \leq n$, we must prove that \emph{all} colorings on $K_n$ have the desired property.

\section{Probabilistic proofs}

In finding Ramsey numbers, one often uses the \textit{probabilistic method} in their proofs. This technique was pioneered by Paul Erdős in \citeyear{erdos1947some}. The key idea is to define a random process that builds a coloring, and show that the probability that some property holds is less than one. From this, it must follow that there exists a coloring where this property does \emph{not} hold.

This argument is non-constructive: it shows that such a counterexample exists, but does not tell us what it is. Despite this caveat, the method is valid and its conclusion is definitely correct.

To illustrate the probabilistic method, we give a proof of the lower bound $R(k,k) \geq 2^{\frac k 2}$. This proof depends on a short lemma, which we will introduce first.

\begin{Lemma} \label{binomial_bound}
    For all positive integers $n$ and $k$,
    \[ {n \choose k} \leq \frac{n^k}{2^{k-1}} \ . \]
\end{Lemma}

\begin{proof}
    Expanding the definition of the binomial coefficient, we see that
    \[
        {n \choose k}
        = \frac{n!}{(n-k)!k!}
        = \frac{n(n-1)\ldots(n-(k-1))}{k!}
        \leq \frac{n^k}{k!} \ .
    \]
    To complete the proof, we use the identity
    \[
        2^{k-1}
        = 1 \times 2 \times 2 \times \ldots \times 2
        \leq 1 \times 2 \times 3 \times \ldots \times k
        = k!
    \]
    to conclude that
    \[
        {n \choose k} \leq \frac{n^k}{2^{k-1}} \ . \qedhere
    \]
\end{proof}

\begin{Theorem}
    For all $k \geq 2$, the following lower bound holds for the Ramsey numbers:
    \[ R(k,k) \geq 2^{\frac k 2} \ . \]
\end{Theorem}

\begin{proof}
    Let $k \geq 2$. Consider the complete graph $K_n$ on $n$ vertices, where $n < 2^{\frac k 2}$. Since every pair of vertices in this graph is connected by an edge, there must be $n \choose 2$ edges in total.

    Now, suppose that each edge is colored red or blue independently with probability $\frac 1 2$. As each of these $2^{n \choose 2}$ colorings is equally likely, the probability of choosing a particular coloring is $2^{-{n \choose 2}}$.

    Let $A$ be a subset of vertices of size $k$. The probability of the event $A_\red$ that all edges within $A$ are colored red is then $2^{-{k \choose 2}}$. Hence it follows that the probability $p_\red$ that \emph{some} $k$-set is colored all red is bounded by
    \[
        p_\red = \Pr(\bigcup_{\abs{A} = k} A_\red)
        \leq \sum_{\abs{A} = k} \Pr(A_\red)
        = {n \choose k} 2^{-{k \choose 2}} \ .
    \]

Now, applying \cref{binomial_bound}, we use the fact that $n < 2^{\frac k 2}$ to conclude that
    \[
        {n \choose k} 2^{-{k \choose 2}}
        \leq \frac{n^k}{2^{k-1}} 2^{-{k \choose 2}}
        < 2^{\frac{k^2}{2} - {k \choose 2} - k + 1}
        = 2^{-\frac k 2 + 1}
        \leq \frac 1 2 \ .
    \]

Hence $p_\red < \frac 1 2$, and by symmetry $p_\blue < \frac 1 2$ for the probability that some $k$ vertices have all edges between them colored blue. It follows that $p_\red + p_\blue < 1$, so there must exist a coloring with no red or blue $K_k$. This means that $K_n$ does not have the Ramsey property for $(k, k)$, and the Ramsey number $R(k,k) \geq 2^{\frac k 2}$.
\end{proof}

\bibliographystyle{abbrvnat}
\bibliography{sources}

\end{document}

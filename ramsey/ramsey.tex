\documentclass{article}
\usepackage[utf8]{inputenc}
\usepackage[T1]{fontenc}
\usepackage{geometry}
\RequirePackage[orthodox]{nag}
\usepackage{microtype}
\usepackage{booktabs}
\usepackage{mathptmx}
\usepackage{mathtools}
\usepackage{amssymb}
\usepackage{amsthm}
\usepackage{stmaryrd}
\usepackage{booktabs}
\usepackage{enumerate}
\usepackage{multirow}
\usepackage{nicefrac}
\usepackage{xspace}

\usepackage{algpseudocode}
\usepackage[colorlinks=true,linkcolor=blue]{hyperref}
\usepackage[colorinlistoftodos]{todonotes}

\allowdisplaybreaks

\theoremstyle{definition}
\newtheorem{Definition}{Definition}

\theoremstyle{plain}
\newtheorem{Thm}{Theorem}
\newtheorem{Conjecture}{Conjecture}
\newtheorem{Corollary}[Thm]{Corollary}
\newtheorem{Example}[Thm]{Example}
\newtheorem{Lemma}[Thm]{Lemma}
\newtheorem{Proposition}[Thm]{Proposition}
\newtheorem{Question}{Question}
\newtheorem{Remark}[Thm]{Remark}
\newtheorem{Theorem}[Thm]{Theorem}

\renewcommand{\mid}{\, \middle| \,}
\newcommand{\abs}[1]{\left| #1 \right|}
\newcommand{\menge}[1]{\ensuremath{\left\{ #1 \right\}}}
\newcommand{\set}[2]{\mbox{$ \left\{ \,#1 \mid #2 \,\right\}$}}
\newcommand{\ext}[1]{\llbracket #1  \rrbracket}
\newcommand{\fa}[2]{\forall {#1} \quad {#2}}
\newcommand{\ex}[2]{\exists {#1} \quad {#2}}

\DeclarePairedDelimiter\floor{\lfloor}{\rfloor}

\newcommand{\N}{\mathbb{N}}
\newcommand{\Q}{\mathbb{Q}}
\newcommand{\R}{\mathbb{R}}

\newcommand{\FCCS}{\textsc{fccs}}

\begin{document}
\author{Chris Wong}
\title{Frank Ramsey's Fantastic Numbers and Where to Find Them}
\maketitle

\todo[inline]{write introduction}

\section{Introduction}

Before we introduce the Ramsey numbers, we must first define a few terms.

A \textit{graph} $G = (V, E)$ consists of a set of \textit{vertices} $V$ connected by a set of \textit{edges} $E$. For the purposes of this paper, we assume that the graphs are \textit{simple}: that is, every pair of vertices has at most one edge between them.

A graph is \textit{complete} when every pair of vertices is connected by an edge. For each positive integer $n$, the complete graph on $n$ vertices is unique and denoted $K_n$.

An \textit{(edge-induced) subgraph} $G' \subseteq G$ is formed from a subset of the edges of $G$, along with the vertices incident to them. When a subgraph is complete, it is called a \textit{clique}.

An \textit{(edge) coloring} labels every edge of a graph by elements from a given set of colors. By tradition, we use the colors \textit{red} and \textit{blue}.

With these definitions, we can state Ramsey's theorem, and with it, the Ramsey numbers:

\begin{Theorem}[Ramsey's theorem] \label{ramseys_theorem}
    Let $s$ and $t$ be positive integers. Then for sufficiently large $n$, every red-blue edge coloring of $K_n$ either has a fully red clique of size $s$, or a fully blue clique of size $t$.
\end{Theorem}

\begin{Definition}[Ramsey numbers]
    Define $R(s,t)$ as the smallest $n$ such that the conclusion of \cref{ramseys_theorem} holds. This is the \textit{Ramsey number} of $(s,t)$.
\end{Definition}

\todo[inline]{include clique/stable set definition}

\section{1-fish, 2-fish, red fish, blue fish}

As with other areas of mathematics, Ramsey numbers have their simple cases: namely, the values of $R(1,t)$ and $R(2,t)$. These cases are easy to state and prove, and so are given below.

\begin{Proposition}
    For any positive $s$ and $t$,
    \[ R(s,1) = R(1,t) = 1 \ . \]
\end{Proposition}

\begin{proof}
    Observe that a clique with one vertex has no internal edges. Hence, trivially, all of its edges are colored both red and blue. The smallest complete graph with a one-vertex clique is $K_1$ itself. The proposition follows.
\end{proof}

\begin{Proposition}
    For any positive $t$,
    \[ R(2,t) = t \ . \]
\end{Proposition}

\begin{proof}
    Consider a coloring of $K_t$. If every edge is colored blue, then the graph has a blue clique of size $t$; and we are done.

    Otherwise, there must exist at least one edge colored red. Take one such edge, along with the two vertices incident to it. This is a red clique of size 2.
\end{proof}

On learning these results, one may wonder: is the expression for $R(3,t)$ as simple as the two before it? The answer is no---the best bounds we have are due to \citet{RSA:RSA3240070302}, who won a Fulkerson prize for his work.

\begin{Theorem}[\citet{RSA:RSA3240070302}] \label{kims_theorem}
    For sufficiently large $t$, the value of $R(3,t)$ can be approximated by $t^2/\log t$.
\end{Theorem}

Finally, we observe that the colors \emph{red} and \emph{blue} are interchangable: any coloring that satisfies the Ramsey property for $(s,t)$ can be adapted to one for $(t,s)$ by swapping the colors.

\begin{Proposition}
    For any positive $s$ and $t$,
    \[ R(s,t) = R(t,s) \ . \]
\end{Proposition}

\begin{proof}
    By symmetry of red and blue.
\end{proof}

\section{Computing Ramsey numbers}

The definition of Ramsey numbers is easy to state. What is surprising, then, is how difficult they are to compute. As of this writing, the last exact result was proven by \cite{JGT:JGT3190190304}, who showed that $R(4,5) = 25$. The value of $R(5,5)$---just one step above---is still an open problem!

\todo[inline]{insert table of known Ramsey numbers}

This difficulty stems from the sheer number of combinations that must be checked. Since a Ramsey number is a statement about \emph{all} colorings of a particular graph, it is not enough to find a single example---we have to account for them all. Moreover, as the number of possible colorings grows exponentially with the number of vertices, it is almost never feasible to check every coloring. In fact, there is no known polynomial-time algorithm for computing Ramsey numbers in general; mathematicians so far have used a separate, ad-hoc proof for each result.

\section{Upper and lower bounds}

While it can be difficult to compute a Ramsey number exactly, we can often derive bounds on what values the number can take. These bounds can be an important step toward finding the exact result. In particular, if we can show that $R(s,t) \geq n$ and $R(s,t) \leq n$ for the same $n$, then it follows that $R(s,t) = n$.

To prove a lower bound for a Ramsey number---that is, $R(s,t) > n$---we only need a counterexample: a coloring on $K_n$ which does \emph{not} have a red clique of size $s$ or a blue clique of size $t$. This coloring is called a \textit{Ramsey graph} of size $n$. Since a Ramsey number is a statement about all colorings on $K_n$, such a coloring is sufficient to show that $R(s,t) \nleq n$, hence $R(s,t) > n$.

On the other hand, to prove an upper bound is often more difficult. This is because to show that $R(s,t) \leq n$, we must prove that \emph{all} colorings on $K_n$ have the desired property.

That is not to say that lower bounds are not hard to find. For all but trivial values for $s$ and $t$, the number of possible colorings is too large to search exhaustively. \todo[inline]{explain how counterexamples are often hard to find too} \todo[inline]{include bounds on $R(5,5)$}

\section{The probabilistic method}

One interesting aspect of Ramsey numbers, as well as graph theory as a whole, is the breadth of techniques used in its proofs. Past practitioners have taken ideas from linear programming and probability theory, and used computational methods to narrow down their results. The approximation $R(3,t) \sim t^2/\log t$, as we saw earlier in \cref{kims_theorem}, was proven using differential equations; the bound $R(4,6) \leq 41$ by \citet{mckay1997subgraph} was shown through linear programming.

For this report, we will focus on one such technique: the \textit{probabilistic method}, pioneered by Paul Erdős in \citeyear{erdos1947some}. The key idea is to define a random process that builds a coloring, and show that the probability that some property holds is less than one. From this, it must follow that there exists a coloring where this property does \emph{not} hold.

This argument is non-constructive: it shows that such a counterexample exists, but does not tell us what it is. Despite this caveat, the method is valid and its conclusion is correct with no chance of error.

To illustrate the probabilistic method, we give a proof of the lower bound $R(k,k) \geq 2^{\frac k 2}$. This proof depends on a short lemma, which we will introduce first.

\begin{Lemma} \label{binomial_bound}
    For all positive integers $n$ and $k$,
    \[ {n \choose k} \leq \frac{n^k}{2^{k-1}} \ . \]
\end{Lemma}

\begin{proof}
    Expanding the definition of the binomial coefficient, we see that
    \[
        {n \choose k}
        = \frac{n!}{(n-k)!k!}
        = \frac{n(n-1)\ldots(n-(k-1))}{k!}
        \leq \frac{n^k}{k!} \ .
    \]
    To complete the proof, we use the identity
    \[
        2^{k-1}
        = 1 \times 2 \times 2 \times \ldots \times 2
        \leq 1 \times 2 \times 3 \times \ldots \times k
        = k!
    \]
    to conclude that
    \[
        {n \choose k} \leq \frac{n^k}{2^{k-1}} \ . \qedhere
    \]
\end{proof}

\begin{Theorem}[\cite{erdos1947some}]
    For all $k \geq 2$, the following lower bound holds for the Ramsey numbers:
    \[ R(k,k) \geq 2^{\frac k 2} \ . \]
\end{Theorem}

\begin{proof}
    Let $k \geq 2$. Consider the complete graph $K_n$ on $n$ vertices, where $n < 2^{\frac k 2}$. Since every pair of vertices in this graph is connected by an edge, there must be $n \choose 2$ edges in total.

    Now, suppose that each edge is colored red or blue independently with probability $\frac 1 2$. As each of these $2^{n \choose 2}$ colorings is equally likely, the probability of choosing a particular coloring is $2^{-{n \choose 2}}$.

    Let $A$ be a subset of vertices of size $k$. The probability of the event $A_\red$ that all edges within $A$ are colored red is then $2^{-{k \choose 2}}$. Hence it follows that the probability $p_\red$ that \emph{some} $k$-set is colored all red is bounded by
    \[
        p_\red = \Pr(\bigcup_{\abs{A} = k} A_\red)
        \leq \sum_{\abs{A} = k} \Pr(A_\red)
        = {n \choose k} 2^{-{k \choose 2}} \ .
    \]

Now, applying \cref{binomial_bound}, we use the fact that $n < 2^{\frac k 2}$ to conclude that
    \[
        {n \choose k} 2^{-{k \choose 2}}
        \leq \frac{n^k}{2^{k-1}} 2^{-{k \choose 2}}
        < 2^{\frac{k^2}{2} - {k \choose 2} - k + 1}
        = 2^{-\frac k 2 + 1}
        \leq \frac 1 2 \ .
    \]

Hence $p_\red < \frac 1 2$, and by symmetry $p_\blue < \frac 1 2$ for the probability that some $k$ vertices have all edges between them colored blue. It follows that $p_\red + p_\blue < 1$, so there must exist a coloring with no red or blue $K_k$. This means that $K_n$ does not have the Ramsey property for $(k, k)$, and the Ramsey number $R(k,k) \geq 2^{\frac k 2}$.
\end{proof}

\section{Conclusion}

\todo[inline]{write conclusion}

\listoftodos

\bibliographystyle{abbrvnat}
\bibliography{sources}

\end{document}

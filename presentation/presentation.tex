\documentclass{beamer}
\usepackage[utf8]{inputenc}
\usepackage[T1]{fontenc}
\usepackage{geometry}
\RequirePackage[orthodox]{nag}
\usepackage{microtype}
\usepackage{booktabs}
\usepackage{mathtools}
\usepackage{amssymb}
\usepackage{amsthm}
\usepackage{stmaryrd}
\usepackage{csquotes}
\usepackage{enumitem}
\usepackage{nicefrac}
\usepackage{todonotes}
\usepackage{xspace}

\newcommand{\fa}[2]{\forall {#1} \quad {#2}}
\newcommand{\ex}[2]{\exists {#1} \quad {#2}}

% http://tex.stackexchange.com/a/23181/36565
\makeatletter
\MHInternalSyntaxOn
\newcommand*\DeclarePairedDelimiterAlt[3]{%
  \@ifdefinable{#1}{
    \@namedef{MT_delim_\MH_cs_to_str:N #1 _star:}##1
        {\mathopen{}\mathclose\bgroup\left#2 ##1 \aftergroup\egroup\right #3}%
    \@xp\@xp\@xp
      \newcommand
        \@xp\csname MT_delim_\MH_cs_to_str:N #1 _nostar:\endcsname
        [2][\\@gobble]
        {
          \mathopen{\@nameuse {\MH_cs_to_str:N ##1 l} #2} ##2
          \mathclose{\@nameuse {\MH_cs_to_str:N ##1 r} #3}}
    \DeclareRobustCommand{#1}{
      \@ifstar
        {\@nameuse{MT_delim_\MH_cs_to_str:N #1 _nostar:}}
        {\@nameuse{MT_delim_\MH_cs_to_str:N #1 _star:}}
    }
  }
}
\MHInternalSyntaxOff
\makeatother

\DeclarePairedDelimiterAlt\abs{\lvert}{\rvert}
\DeclarePairedDelimiterAlt\ceil{\lceil}{\rceil}
\DeclarePairedDelimiterAlt\floor{\lfloor}{\rfloor}
\DeclarePairedDelimiterAlt\menge{\{}{\}}

\newcommand{\N}{\mathbb{N}}
\newcommand{\Q}{\mathbb{Q}}
\newcommand{\R}{\mathbb{R}}

\begin{document}
\title{Computable analysis with fast converging Cauchy sequences}
\author{Chris Wong}
\institute{University of Canterbury}

\frame{\titlepage}

\setlength{\parskip}{\baselineskip}

\section{Introduction}
\begin{frame}
    \frametitle{Two problems}

    A \textbf{practical} problem,

    \vfill

    \hfill \ldots and a \textbf{philosophical} one.

\end{frame}

\begin{frame}
    \frametitle{Floating point}

    \[
        \begin{pmatrix}
            64919121 & -159018721 \\
            41869520.5 & -102558961
        \end{pmatrix}
        \begin{pmatrix}
            x \\ y
        \end{pmatrix}
        =
        \begin{pmatrix}
            1 \\ 0
        \end{pmatrix}
    \]

\end{frame}

\begin{frame}
    \frametitle{Solution!}

    \[
        \begin{pmatrix}
            x \\ y
        \end{pmatrix}
        =
        \begin{pmatrix}
            102558961 \\ 41869520.5
        \end{pmatrix}
    \]

\end{frame}

\begin{frame}
    \frametitle{Solution\ldots?}

    Computed ``solution''
    \[
        \begin{pmatrix}
            x \\ y
        \end{pmatrix}
        =
        \begin{pmatrix}
            102558961 \\ 41869520.5
        \end{pmatrix}
    \]

    Actual solution
    \[
        \begin{pmatrix}
            x \\ y
        \end{pmatrix}
        =
        \begin{pmatrix}
            205117922 \\ 83739041
        \end{pmatrix}
    \]

\end{frame}

\begin{frame}
    \begin{center}
        \includegraphics[height=0.9\paperheight]{atomic_annie_1.jpg}
    \end{center}
\end{frame}

\begin{frame}
    \frametitle{(Mathematical) constructivism}

    \textbf{Constructivism} is the belief that to prove an object exists, we must provide a procedure to construct it

    Constructivists reject the excluded middle ($A \vee \neg A$) and double negation elimination ($\neg\neg A \rightarrow A$) laws

    Reaction against a perceived ``lack of meaning'' in modern mathematics
\end{frame}

\begin{frame}
    The pure mathematician is isolated\dots He suffers from an alienation which is seemingly inevitable; he has followed the gleam and it has led him out of this world.

    \hfill ---Bishop and Bridges, \textit{Constructive Analysis}, 1987
\end{frame}

\begin{frame}
    \frametitle{Construction $\sim$ Computation}

    Since constructive proofs describe procedures, they can be written as computer programs

    Called the ``Curry--Howard correspondence'' or ``BHK interpretation''

    So \textbf{constructive analysis} is also \textbf{computable}
\end{frame}

\begin{frame}
    \frametitle{Fast converging Cauchy sequences}
    \begin{Definition}
        Let $x : \N \rightarrow \Q$ be a sequence of rational numbers.

        If for any natural numbers $n$ and $m$,
        \[
            \abs{x_n - x_{n+m}} < \frac 1 {2^n} \ ,
        \]
        then $x$ is a \textbf{fast converging Cauchy sequence}.
    \end{Definition}

    Intuition: each element is at most $1/2^n$ away from the limit point.
\end{frame}

\begin{frame}
    \frametitle{Addition and subtraction}

    When we add two sequences pointwise, we also add the errors

    So the error of the result is at most \emph{double} the error of the inputs

    Solution: shift the sequence by one step, halving the error to compensate
\end{frame}

\begin{frame}
    \frametitle{Addition and subtraction}

    \begin{Example}
        Let $x := (2, 1\frac 1 2, 1\frac 1 4, 1\frac 1 8, \dots)$.

        Let $y := (3, 2\frac 1 2, 2\frac 1 4, 2\frac 1 8, \dots)$.

        Adding pointwise, we get $z = (5, 4, 3\frac 1 2, 3\frac 1 4, \dots)$.

        This is not fast converging because
        $\abs{z_0 - z_2} = 1\frac 1 2 \nless \frac 1 {2^0}$.

        Shift $z$ to get $z' = (4, 3\frac 1 2, 3\frac 1 4, \dots)$ which is fast converging.
    \end{Example}
\end{frame}

\begin{frame}
    \frametitle{Multiplication}
    With multiplication, the shift depends on the magnitude of the inputs

    For example, if $x = 16$ then we need to shift by $\log_2 16 = 4$ steps

    (Details in thesis)
\end{frame}

\begin{frame}
    \frametitle{Division?}
    We'll come back to that!
\end{frame}

\begin{frame}
    \frametitle{Equality}

    \begin{Definition}
        A pair of FCCS $x$ and $y$ are \textbf{equal} when for every natural number $n$,
        \[ \abs{x_{n+1} - y_{n+1}} \leq \frac 1 {2^n} \ . \]
    \end{Definition}

    Note the use of $n+1$ as the index!
\end{frame}

\begin{frame}
    \frametitle{Equality is undecidable}
    \begin{Theorem}
        The proposition
        \[ x = y \vee \neg (x = y) \]
        is not derivable for arbitrary $x$ and $y$.
    \end{Theorem}
    \begin{proof}
        Given an arbitrary program $P$, define a FCCS $x$ which converges to zero if and only if $P$ does not terminate. By checking if $x = 0$, we can solve the Halting Problem. Contradiction.
    \end{proof}
\end{frame}

\begin{frame}
    \frametitle{Inequality and apartness}
    Constructive analysis makes a distinction between \emph{non-constructive} (not-equal) and \emph{constructive} inequality (apart):
    \begin{description}
        \item[Not equal] Not all elements are close enough
        \item[Apart] We \emph{know} the index after which subsequent elements will never come closer together
    \end{description}
    $\textrm{A} \Rightarrow \textrm{NE}$ always, but $\textrm{NE} \Rightarrow \textrm{A}$ requires \textbf{Markov's principle}.

    Under classical logic, Markov's principle is trivial and so the two notions are equivalent.
\end{frame}

\begin{frame}
    \frametitle{Division}
    The reciprocal of $x$ is only defined when $x \neq 0$

    In fact, evidence of apartness from 0 gives a lower bound on the size of $x$

    This bound tells us how much we need to shift $x^{-1}$
\end{frame}

\begin{frame}
    \frametitle{Exponentials and trigonometric functions}
    Use power series expansion
    \[
        \exp(x) = \sum_{n=0}^\infty = 1 + x + \frac{x}{2}
        + \frac{x^3}{3!} + \frac{x^4}{4!} + \dots
    \]
    Observe that when $-1 \leq x \leq 1$, the numerator of each term is smaller than 1

    Strategy: prove fast convergence on $[-1,1]$, then use identity $\exp(2x) = \exp(x)^2$ to extend it to all of $\Q$

    Can define $\sin$ and $\cos$ in a similar way

    Question: what if $x$ is a FCCS?
\end{frame}

\begin{frame}
    \frametitle{Cantor's diagonal argument}
    \begin{Theorem}
        Let $x : \N \rightarrow \mathrm{FCCS}$ be a sequence of fast converging Cauchy sequences. Then we can construct a FCCS $\mathrm{cantor}(x)$ that is apart from every element in $x$.
    \end{Theorem}

    Analogous to argument on $\R$

    Shows that the set of computable reals (FCCS) is \textbf{countable} but not \textbf{enumerable}
\end{frame}

\begin{frame}
    \frametitle{Thanks for listening!}
    \begin{center}
        \includegraphics[height=0.5\paperwidth]{Lucerne_flowers.jpg}

        \url{https://github.com/lambda-fairy/plusone}
    \end{center}
\end{frame}

\end{document}

\documentclass{beamer}
\usepackage[utf8]{inputenc}
\usepackage[T1]{fontenc}
\usepackage{geometry}
\RequirePackage[orthodox]{nag}
\usepackage{microtype}
\usepackage{booktabs}
\usepackage{mathtools}
\usepackage{amssymb}
\usepackage{amsthm}
\usepackage{stmaryrd}
\usepackage{csquotes}
\usepackage{enumitem}
\usepackage{nicefrac}
\usepackage{todonotes}
\usepackage{xspace}

\newcommand{\fa}[2]{\forall {#1} \quad {#2}}
\newcommand{\ex}[2]{\exists {#1} \quad {#2}}

% http://tex.stackexchange.com/a/23181/36565
\makeatletter
\MHInternalSyntaxOn
\newcommand*\DeclarePairedDelimiterAlt[3]{%
  \@ifdefinable{#1}{
    \@namedef{MT_delim_\MH_cs_to_str:N #1 _star:}##1
        {\mathopen{}\mathclose\bgroup\left#2 ##1 \aftergroup\egroup\right #3}%
    \@xp\@xp\@xp
      \newcommand
        \@xp\csname MT_delim_\MH_cs_to_str:N #1 _nostar:\endcsname
        [2][\\@gobble]
        {
          \mathopen{\@nameuse {\MH_cs_to_str:N ##1 l} #2} ##2
          \mathclose{\@nameuse {\MH_cs_to_str:N ##1 r} #3}}
    \DeclareRobustCommand{#1}{
      \@ifstar
        {\@nameuse{MT_delim_\MH_cs_to_str:N #1 _nostar:}}
        {\@nameuse{MT_delim_\MH_cs_to_str:N #1 _star:}}
    }
  }
}
\MHInternalSyntaxOff
\makeatother

\DeclarePairedDelimiterAlt\abs{\lvert}{\rvert}
\DeclarePairedDelimiterAlt\ceil{\lceil}{\rceil}
\DeclarePairedDelimiterAlt\floor{\lfloor}{\rfloor}
\DeclarePairedDelimiterAlt\menge{\{}{\}}

\newcommand{\N}{\mathbb{N}}
\newcommand{\Q}{\mathbb{Q}}
\newcommand{\R}{\mathbb{R}}

\begin{document}
\title{Computable analysis with fast converging Cauchy sequences}
\author{Chris Wong}
\institute{University of Canterbury}

\frame{\titlepage}

\section{Introduction}
\begin{frame}
    \frametitle{Two problems}

    A \textbf{practical} problem,

    \vfill

    \hfill \ldots and a \textbf{philosophical} one.

\end{frame}

\begin{frame}
    \frametitle{Floating point}

    \[
        \begin{pmatrix}
            64919121 & -159018721 \\
            41869520.5 & -102558961
        \end{pmatrix}
        \begin{pmatrix}
            x \\ y
        \end{pmatrix}
        =
        \begin{pmatrix}
            1 \\ 0
        \end{pmatrix}
    \]

\end{frame}

\begin{frame}
    \frametitle{Solution!}

    \[
        \begin{pmatrix}
            x \\ y
        \end{pmatrix}
        =
        \begin{pmatrix}
            102558961 \\ 41869520.5
        \end{pmatrix}
    \]

\end{frame}

\begin{frame}
    \frametitle{Solution\ldots?}

    Computed ``solution''
    \[
        \begin{pmatrix}
            x \\ y
        \end{pmatrix}
        =
        \begin{pmatrix}
            102558961 \\ 41869520.5
        \end{pmatrix}
    \]

    Actual solution
    \[
        \begin{pmatrix}
            x \\ y
        \end{pmatrix}
        =
        \begin{pmatrix}
            205117922 \\ 83739041
        \end{pmatrix}
    \]

\end{frame}

\begin{frame}
    \begin{center}
        \includegraphics[height=0.9\paperheight]{Hindenburg_disaster,_1937.jpg}
    \end{center}
\end{frame}

\end{document}
